\documentclass{jsarticle}
\usepackage{fenrir_v1_4_0}
\usepackage{autocount}
\usepackage{ethm_v1_1_0}
\mathtoolsset{showonlyrefs=true}

\begin{document}
\textbf{Exercise 5.33 (Study of multidimensional Brownian motion)}
$B_t=(B_t^{1},B_t^{2},\dotsb,B_t^{N})$ を $x=(x_1,\dotsb,x_N)(\in\real^N)$ スタートの $N$ 次元 $(\clf_t)$-BMとする.
ここで $N$ は2以上の整数とする.
\begin{enumerate}
    \item
    $\abs{B_t}^2$ は連続semimartingaleであり,$\abs{B_t}^2$ のmartingale partがtrue martingaleであることを示せ.
    \begin{proof}\textcolor{red}{(途中)}
        $B_t^{1},\dotsb,B_t^{N}$ はBMより連続semimartingaleなので,伊藤の公式が適用できて,a.s.で任意の $t\ge0$ に対し
        \begin{align}
            \abs{B_t}^2
            &= \abs{B_0}^2
            + \sum_{i=1}^{N}\int_{0}^{t}\frac{\partial}{\partial x_i}\abs{B_s}^2 dB_s^i
            + \frac{1}{2}\sum_{i,j=1}^{N}\int_{0}^{t}\frac{\partial^2}{\partial x_i\partial x_j}\abs{B_s}^2 d\gen{B^i,B^j}_s \\
            &= \abs{x}^2
            + \sum_{i=1}^{N}\int_{0}^{t}\frac{\partial}{\partial x_i}\abs{B_s}^2 dB_s^i
            + \frac{1}{2}\sum_{i=1}^{N}\int_{0}^{t}\frac{\partial^2}{\partial x_i^2}\abs{B_s}^2 ds
            \quad(i\neq j\implies\gen{B^i,B^j}=0)\\
            &= \abs{x}^2
            + \sum_{i=1}^{N}\int_{0}^{t}2B_s^i dB_s^i
            + Nt.
        \end{align}
        \begin{screen}
            $\because)$ $\abs{B}^2=(B^1)^2+\dotsb+(B^N)^2$ より $\frac{\partial}{\partial x_i}\abs{B}^2=2B^i, \frac{\partial^2}{\partial x_i^2}\abs{B}^2=2.$
        \end{screen}
    \end{proof}
    
    \item
    \begin{align}
        \beta_t
        = \sum_{i=1}^{N}\int_{0}^{t}\frac{B_{s}^{i}}{\abs{B_s}}dB_s^i
    \end{align}
    と定める(ただし $\abs{B_s}=0$ のとき $\frac{B_{s}^{i}}{\abs{B_s}}=0$ とする).
    $\beta_t$ の定義に現れる確率積分の定義を正当化し,さらに $(\beta_t)_{t\ge0}$ が0スタートの $(\clf_t)$-BMであることを示せ.
    \begin{proof}\textcolor{red}{(途中)}
        
    \end{proof}
    
    \item
    $$
    \abs{B_t}^2
    = \abs{x}^2+2\int_0^t \abs{B_t}d\beta_s+Nt
    $$
    が成り立つことを示せ.
    \begin{proof}\textcolor{red}{(途中)}
        
    \end{proof}
    
    \item
    以降,$x\neq0$ を仮定する.
    $\ep\in(0,\abs{x}),T_{\ep}=\inf\set{t\ge0}{\abs{B_t}\le\ep}$ とする.
    ここで任意の $a>0$ に対し
    $$
    f(a)=
    \begin{cases}
        \log a & (N=2), \\
        a^{2-N} & (N\ge3)
    \end{cases}
    $$
    と定める.
    $f(\abs{B_{t\wedge T_{\ep}}})$ がCLMとなることを示せ.
    \begin{proof}\textcolor{red}{(途中)}
        
    \end{proof}
    
    \item
    $R>\abs{x},S_R=\inf\set{t\ge0}{\abs{B_t}\ge R}$ とする.
    $$
    P(T_\ep<S_R)=\frac{f(R)-f(\abs{x})}{f(R)-f(\ep)}
    $$
    となることを示せ.
    また $\ep\to0$ としたとき $P(T_\ep<S_R)\to0$ となることを確かめ,a.s.で任意の $t\ge0$ に対し $B_t\neq0$ となることを示せ.
    \begin{proof}\textcolor{red}{(途中)}
        
    \end{proof}
    
    \item
    a.s.で任意の $t\ge0$ に対し
    $$
    \abs{B_t}
    = \abs{x}+\beta_t+\frac{N-1}{2}\int_0^t \frac{ds}{\abs{B_s}}
    $$
    となることを示せ.
    \begin{proof}\textcolor{red}{(途中)}
        
    \end{proof}
    
    \item
    $N\ge3$を仮定する.
    a.s.で$t\to\infty$ としたとき $\abs{B_t}\to\infty$ となることを示せ(ヒント:$\abs{B_t}^{2-N}$ が非負supermartingaleであることを確かめよ).
    \begin{proof}\textcolor{red}{(途中)}
        
    \end{proof}
    
    \item
    $N=3$ を仮定する.
    Gaussian densityの形式を用いて,r.v.の族 $(\abs{B_t}^{-1})_{t\ge0}$ が $L^2$-bdd.であることを確かめよ.
    また $(\abs{B_t}^{-1})_{t\ge0}$ がCLMであり,かつtrue martingaleでないことを示せ.
    \begin{proof}\textcolor{red}{(途中)}
        
    \end{proof}
\end{enumerate}
\end{document}