\documentclass{jsarticle}
\usepackage{fenrir_v1_4_0}
\usepackage{autocount}
\usepackage{ethm_v1_1_0}
\mathtoolsset{showonlyrefs=true}

\begin{document}
\textbf{Exercise 5.33 (Study of multidimensional Brownian motion)}
$B_t=(B_t^{1},B_t^{2},\dotsb,B_t^{N})$ を $x=(x_1,\dotsb,x_N)(\in\real^N)$ スタートの $N$ 次元 $(\clf_t)$-BMとする.
ここで $N$ は2以上の整数とする.
\begin{enumerate}
    \item
    $\abs{B_t}^2$ は連続semimartingaleであり,$\abs{B_t}^2$ のmartingale partがtrue martingaleであることを示せ.
    \begin{proof}
        $B_t^{1},\dotsb,B_t^{N}$ はBMより連続semimartingale.
        よって $F(x)=\abs{x}^2$ に対してIt\^{o}'s formulaが適用できて,a.s.で任意の $t\ge0$ に対し
        \begin{align}
            \abs{B_t}^2
            = F(B_t)
            &= \abs{B_0}^2
            + \sum_{i=1}^{N}\int_{0}^{t}\frac{\partial}{\partial x_i}\abs{B_s}^2 dB_s^i
            + \frac{1}{2}\sum_{i,j=1}^{N}\int_{0}^{t}\frac{\partial^2}{\partial x_i\partial x_j}\abs{B_s}^2 d\gen{B^i,B^j}_s \\
            &= \abs{x}^2
            + \sum_{i=1}^{N}\int_{0}^{t}\frac{\partial}{\partial x_i}\abs{B_s}^2 dB_s^i
            + \frac{1}{2}\sum_{i=1}^{N}\int_{0}^{t}\frac{\partial^2}{\partial x_i^2}\abs{B_s}^2 ds
            \quad(\because i\neq j\implies\gen{B^i,B^j}=0)\\
            &= \abs{x}^2
            + \sum_{i=1}^{N}\int_{0}^{t}2B_s^i dB_s^i
            + Nt.
        \end{align}
        \begin{screen}
            $\because)$ $\abs{B}^2=\sum_{i=1}^{N}(B^i)^2$ より $\frac{\partial F}{\partial x_i}(B)=2B^i, \frac{\partial^2 F}{\partial x_i^2}(B)=2.$
        \end{screen}

        a.s.で任意の $t\ge0$ に対し
        $$
        \int_{0}^{t}(2B_s^i)^2 ds
        \le 4t\sup_{0\le s\le t}(B_s^i)^2
        < \infty
        $$
        が成り立つので,任意の $1\le i\le N$ に対し $2B^i\in L_{\mathrm{loc}}^2(B^i).$
        したがって $(\sum_{i=1}^{N}\int_{0}^{t}2B_s^i dB_s^i)_{t\ge0}$ は CLM.
        また $(\abs{x}^2+Nt)_{t\ge0}$ はFVなので,$\abs{B}^2$ は連続semimartingaleである.

        次に $(\sum_{i=1}^{N}\int_{0}^{t}2B_s^i dB_s^i)_{t\ge0}$ がtrue martingaleであることを示す.
        Doob's ineq. in $L^2$ より
        \begin{align}
            E[\gen{\sum_{i=1}^{N}\int_{0}^{\cdot}2B_s^i dB_s^i, \sum_{i=1}^{N}\int_{0}^{\cdot}2B_s^i dB_s^i}_t]
            &= E[\sum_{i=1}^{N}\int_{0}^{t}(2B_s^i)^2 ds] \\
            &\le 4t\sum_{i=1}^{N}E[\sup_{0\le s\le t}(B_s^i)^2] \\
            &\le 4t2^2\sum_{i=1}^{N}E[(B_s^i)^2] \\
            &= 16t(\abs{x}^2+Nt).
        \end{align}
        \begin{screen}
            $\because)$
            任意の $1\le i\le N$ に対し
            \begin{align}
                t
                &= E[(B_t^i-x_i)^2] \\
                &= E[(B_t^i)^2]
                - 2x_iE[B_t^i]
                + E[x_i^2] \\
                &= E[(B_t^i)^2] - x_i^2
            \end{align}
            より $E[(B_t^i)^2]=x_i^2+t.$
        \end{screen}

        よってThm. 4.13(ii)より $(\sum_{i=1}^{N}\int_{0}^{t}2B_s^i dB_s^i)_{t\ge0}$ はtrue martingale.
    \end{proof}
    
    \item
    \begin{align}
        \beta_t
        = \sum_{i=1}^{N}\int_{0}^{t}\frac{B_{s}^{i}}{\abs{B_s}}dB_s^i
    \end{align}
    と定める(ただし $\abs{B_s}=0$ のとき $\frac{B_{s}^{i}}{\abs{B_s}}=0$ とする).
    $\beta_t$ の定義に現れる確率積分の定義を正当化し,さらに $(\beta_t)_{t\ge0}$ が0スタートの $(\clf_t)$-BMであることを示せ.
    \begin{proof}
        任意の $1\le i\le N$ に対し $\frac{B^i}{\abs{B}}\le1$ より,a.s.で任意の $t\ge0$ に対し
        \begin{align}
            \int_0^t \left(\frac{B_s^i}{\abs{B_s}}\right)^2 ds
            \le \int_0^t ds = t < \infty
        \end{align}
        が成り立つので,任意の $1\le i\le N$ に対し $\frac{B^i}{\abs{B}}\in L_{\mathrm{loc}}^2(B^i).$
        よってThm 5.6より $(\int_{0}^{t}\frac{B_{s}^{i}}{\abs{B_s}}dB_s^i)_{t\ge0}$ は確率積分の意味でwell-definedなCLMである.

        したがって $\beta$ は $(\clf_t)$-adaptedなCLMである.
        ここで
        \begin{align}
            \gen{\beta, \beta}_t
            &= \gen{\sum_{i=1}^{N}\int_{0}^{\cdot}\frac{B_{s}^{i}}{\abs{B_s}}dB_s^i, \sum_{i=1}^{N}\int_{0}^{\cdot}\frac{B_{s}^{i}}{\abs{B_s}}dB_s^i}_t \\
            &= \sum_{i=1}^{N}\gen{\int_{0}^{\cdot}\frac{B_{s}^{i}}{\abs{B_s}}dB_s^i, \int_{0}^{\cdot}\frac{B_{s}^{i}}{\abs{B_s}}dB_s^i}_t \\
            &= \int_{0}^{t}\frac{\sum_{i=1}^{N}(B_{s}^{i})^2}{\abs{B_s}^2}ds
            = \int_{0}^{t}\frac{\abs{B_s}^2}{\abs{B_s}^2}ds = t
        \end{align}
        が成り立つことより,Thm. 5.12から $\beta$ は0スタートの $(\clf_t)$-BMである.
    \end{proof}
    
    \item
    $$
    \abs{B_t}^2
    = \abs{x}^2+2\int_0^t \abs{B_s}d\beta_s+Nt
    $$
    が成り立つことを示せ.
    \begin{proof}
        \begin{align}
            \frac{d\beta_t}{dB_t^i}
            = \sum_{i=1}^{N}\frac{d}{dB_t^i}\int_0^t\frac{B_{s}^{i}}{\abs{B_s}}dB_s^i
            = \sum_{i=1}^{N}\frac{B_{t}^{i}}{\abs{B_t}}
        \end{align}
        より $d\beta_t=\sum_{i=1}^{N}\frac{B_{t}^{i}}{\abs{B_t}}dB_t^i$ となるので
        \begin{align}
            \abs{B_t}^2
            &= \abs{x}^2
            + \sum_{i=1}^{N}\int_{0}^{t}2B_s^i dB_s^i
            + Nt \\
            &= \abs{x}^2
            + \sum_{i=1}^{N}\int_{0}^{t}2\abs{B_s}\frac{B_{s}^{i}}{\abs{B_s}}dB_s^i
            + Nt \\
            &= \abs{x}^2
            + 2\int_{0}^{t}\abs{B_s}\sum_{i=1}^{N}\frac{B_{s}^{i}}{\abs{B_s}}dB_s^i
            + Nt \\
            &= \abs{x}^2
            + 2\int_{0}^{t}\abs{B_s}d\beta_s
            + Nt.
        \end{align}
    \end{proof}
    
    \item
    以降,$x\neq0$ を仮定する.
    $\ep\in(0,\abs{x}),T_{\ep}=\inf\set{t\ge0}{\abs{B_t}\le\ep}$ とする.
    ここで任意の $a>0$ に対し
    $$
    f(a)=
    \begin{cases}
        \log a & (N=2), \\
        a^{2-N} & (N\ge3)
    \end{cases}
    $$
    と定める.
    $f(\abs{B_{t\wedge T_{\ep}}})$ がCLMとなることを示せ.
    \begin{proof}
        $F(x)=f(\abs{x})$ と定めると $F\in C^\infty(\real^N\setminus\{0\})$ であり,
        $$
        \frac{\partial F}{\partial x_i}(x)=
        \begin{cases}
            \frac{x_i}{\abs{x}^2} & N=2, \\
            \frac{(2-N)x_i}{\abs{x}^N} & N\ge3,
        \end{cases}\qquad
        \frac{\partial^2 F}{\partial x_i^2}(x)=
        \begin{cases}
            \frac{1}{\abs{x}^2}\left(1-\frac{2x_i^2}{\abs{x}^2}\right) & N=2, \\
            \frac{2-N}{\abs{x}^N}\left(1-\frac{Nx_i^2}{\abs{x}^2}\right) & N\ge3.
        \end{cases}
        $$

        ここで,任意の $t\ge0,\omega\in\Omega$ に対し $\abs{B_{t\wedge T_\ep}(\omega)}\ge\ep$ が成り立つので,$B_{t\wedge T_\ep}\in\real^N\setminus\{0\}.$
        It\^{o}'s formulaより
        \begin{align}
            & f(\abs{B_{t\wedge T_\ep}})
            = F(B_{t\wedge T_\ep}) \\
            &= 
            \begin{cases}
                f(\abs{x})
                + \sum_{i=1}^{2}\int_0^t\frac{B_{s\wedge T_\ep}^i}{\abs{B_{s\wedge T_\ep}}^2}dB_s^i
                + \frac{1}{2}\sum_{i=1}^{2}\int_0^t\frac{1}{\abs{B_{s\wedge T_\ep}}^2}\left(1-\frac{2(B_{s\wedge T_\ep}^i)^2}{\abs{B_{s\wedge T_\ep}}^2}\right)ds & N=2, \\
                f(\abs{x})
                + \sum_{i=1}^{N}\int_0^t\frac{(2-N)B_{s\wedge T_\ep}^i}{\abs{B_{s\wedge T_\ep}}^N}dB_s^i
                + \frac{1}{2}\sum_{i=1}^{N}\int_0^t\frac{2-N}{\abs{B_{s\wedge T_\ep}}^N}\left(1-\frac{N(B_{s\wedge T_\ep}^i)^2}{\abs{B_{s\wedge T_\ep}}^2}\right)ds & N\ge3
            \end{cases} \\
            &= 
            \begin{cases}
                f(\abs{x})
                + \sum_{i=1}^{2}\int_0^t\frac{B_{s\wedge T_\ep}^i}{\abs{B_{s\wedge T_\ep}}^2}dB_s^i
                + \frac{1}{2}\int_0^t\frac{1}{\abs{B_{s\wedge T_\ep}}^2}\left(2-2\sum_{i=1}^{2}\frac{(B_{s\wedge T_\ep}^i)^2}{\abs{B_{s\wedge T_\ep}}^2}\right)ds & N=2, \\
                f(\abs{x})
                + \sum_{i=1}^{N}\int_0^t\frac{(2-N)B_{s\wedge T_\ep}^i}{\abs{B_{s\wedge T_\ep}}^N}dB_s^i
                + \frac{1}{2}\sum_{i=1}^{N}\int_0^t\frac{2-N}{\abs{B_{s\wedge T_\ep}}^N}\left(N-N\sum_{i=1}^{N}\frac{(B_{s\wedge T_\ep}^i)^2}{\abs{B_{s\wedge T_\ep}}^2}\right)ds & N\ge3
            \end{cases} \\
            &= 
            \begin{cases}
                f(\abs{x})
                + \sum_{i=1}^{2}\int_0^t\frac{B_{s\wedge T_\ep}^i}{\abs{B_{s\wedge T_\ep}}^2}dB_s^i & N=2, \\
                f(\abs{x})
                + \sum_{i=1}^{N}\int_0^t\frac{(2-N)B_{s\wedge T_\ep}^i}{\abs{B_{s\wedge T_\ep}}^N}dB_s^i & N\ge3
            \end{cases}
        \end{align}
        が成り立ち,任意の $N\ge2$ と $1\le i\le N$ に対し $\frac{B_{t\wedge T_\ep}^i}{\abs{B_{t\wedge T_\ep}}^N}\in L_{\mathrm{loc}}^2(B_t^i)$ であるので $\int_0^t\frac{B_{s\wedge T_\ep}^i}{\abs{B_{s\wedge T_\ep}}^N}dB_s^i$ はCLM.
        したがって $f(\abs{B_{t\wedge T_\ep}})$ はCLM.
    \end{proof}
    
    \item
    $R>\abs{x},S_R=\inf\set{t\ge0}{\abs{B_t}\ge R}$ とする.
    $$
    P(T_\ep<S_R)=\frac{f(R)-f(\abs{x})}{f(R)-f(\ep)}
    $$
    となることを示せ.
    また $\ep\to0$ としたとき $P(T_\ep<S_R)\to0$ となることを確かめ,a.s.で任意の $t\ge0$ に対し $B_t\neq0$ となることを示せ.
    \begin{proof}
        $T:=T_\ep\wedge S_R$ とするとa.s.で任意の $t\ge0$ に対し $f(\abs{B_t^T})\le R.$
        $f(\abs{B_{t\wedge T_\ep}})$ がCLMであることから $f(\abs{B_t^T})$ は有界なCLMなので,Prop. 4.7(ii)よりUIM.
        ゆえにoptional stopping theoremより
        \begin{align}
            f(\abs{x})
            &= E[f(\abs{B_0^T})]
            = E[f(\abs{B_T})] \\
            &= E[f(\ep\bm{1}_{\{T_\ep<S_R\}}+R\bm{1}_{\{T_\ep\ge S_R\}})] \\
            &= E[f(\ep)\bm{1}_{\{T_\ep<S_R\}}+f(R)\bm{1}_{\{T_\ep\ge S_R\}}] \\
            &= f(\ep)P(T_\ep<S_R)+f(R)P(T_\ep\ge S_R).
        \end{align}

        $P(T_\ep<S_R)+P(T_\ep\ge S_R)=1$ より $P(T_\ep<S_R)=\frac{f(R)-f(\abs{x})}{f(R)-f(\ep)}$ を得る.

        $\ep\to0$ とすると $N=2$ のとき $f(\ep)\to -\infty, N\ge3$ のとき $f(\ep)\to +\infty$ より,$P(T_\ep<S_R)\to0$ となることがわかる.

        次にa.s.で任意の $t\ge0$ に対し $B_t\neq0$ を示す.
        $\ep_n\downarrow0$ かつ $\sum_{n=1}^{\infty}P(T_{\ep_n}<S_n)<\infty$ を満たす正の実数列 $\{\ep_n\}$ を選ぶと,Borel-Cantelliの補題より $A:=\bigcap_{m\ge1}\bigcup_{n\ge m}\{T_{\ep_n}<S_n\}$ に対し $P(A)=0.$
        このとき任意の $\omega\in A^c, t\ge0$ に対し $B_t(\omega)\neq0.$

        \begin{screen}
            $\because)$
            $\omega\in A^c$ に対し $B_t(\omega)=0$ を満たす $t>0$ が存在すると仮定すると,任意の $n\ge1$ に対し $T_{\ep_n}(\omega)<t$ であり,$A^c$ の定め方からある $m\ge1$ が存在して,任意の $n\ge m$ に対し $S_n(\omega)<t.$
            stopping timeの列 $(S_n(\omega))_{n\ge1}$ はnondecreasingなので,$\lim_{n\to\infty}S_n(\omega)(\le t)$ が存在する.
            これを $s$ とすると $B_s(\omega)=\infty$ となるが,$B$ の連続性より矛盾.
        \end{screen}

        以上よりa.s.で任意の $t\ge0$ に対し $B_t\neq0.$
    \end{proof}
    
    \item
    a.s.で任意の $t\ge0$ に対し
    $$
    \abs{B_t}
    = \abs{x}+\beta_t+\frac{N-1}{2}\int_0^t \frac{ds}{\abs{B_s}}
    $$
    となることを示せ.
    \begin{proof}
        $F(x)=\abs{x}$ と定めると $F\in C^\infty(\real^N\setminus\{0\})$ であり,$\frac{\partial F}{\partial x_i}(x)=\frac{x_i}{\abs{x}}, \frac{\partial^2 F}{\partial x_i^2}(x)=\frac{\abs{x}^2-x_i^2}{\abs{x}^3}.$
        a.s.で任意の $t\ge0$ に対し $B_t\in\real^N\setminus\{0\}$ より,It\^{o}'s formulaから
        \begin{align}
            \abs{B_t}
            = F(B_t)
            &= \abs{x}
            + \sum_{i=1}^{N}\int_0^t \frac{B_s^i}{\abs{B_s}}dB_s^i
            + \frac{1}{2}\sum_{i=1}^{N}\int_0^t \frac{\abs{B_s}^2-(B_s^i)^2}{\abs{B_s}^3}ds \\
            &= \abs{x}
            + \beta_t
            + \frac{1}{2}\int_0^t \frac{N\abs{B_s}^2-\sum_{i=1}^{N}(B_s^i)^2}{\abs{B_s}^3}ds \\
            &= \abs{x}
            + \beta_t
            + \frac{1}{2}\int_0^t \frac{(N-1)\abs{B_s}^2}{\abs{B_s}^3}ds \\
            &= \abs{x}
            + \beta_t
            + \frac{N-1}{2}\int_0^t \frac{ds}{\abs{B_s}}.
        \end{align}
    \end{proof}
    
    \item
    $N\ge3$を仮定する.
    a.s.で$t\to\infty$ としたとき $\abs{B_t}\to\infty$ となることを示せ(ヒント:$\abs{B_t}^{2-N}$ が非負supermartingaleであることを確かめよ).
    \begin{proof}
        $F(x)=\abs{x}^{2-N}$ と定めると $F\in C^\infty(\real^N\setminus\{0\})$ であり,$\frac{\partial F}{\partial x_i}(x)=\frac{(2-N)x_i}{\abs{x}^N}, \frac{\partial^2 F}{\partial x_i^2}(x)=\frac{2-N}{\abs{x}^N}\left(1-\frac{Nx_i^2}{\abs{x}^2}\right).$
        a.s.で任意の $t\ge0$ に対し $B_t\in\real^N\setminus\{0\}$ より,It\^{o}'s formulaと4.の証明から
        \begin{align}
            \abs{B_t}^{2-N}
            = F(B_t)
            &= \abs{x}^{2-N}
            + \sum_{i=1}^{N}\int_0^t \frac{(2-N)B_s^i}{\abs{B_s}^N}dB_s^i
            + \frac{1}{2}\sum_{i=1}^{N}\int_0^t \frac{2-N}{\abs{B_s}^N}\left(1-\frac{N(B_s^i)^2}{\abs{B_s}^2}\right)ds \\
            &= \abs{x}^{2-N}
            + \sum_{i=1}^{N}\int_0^t \frac{(2-N)B_s^i}{\abs{B_s}^N}dB_s^i.
        \end{align}

        4.の結果より $\abs{B}^{2-N}$ は非負CLMなので,Prop. 4.7(i)より$\abs{B}^{2-N}$ は非負supermartingale.
        ゆえに任意の $t\ge0$ に対し
        $$
        E[\abs{B_t}^{2-N}]
        \le E[\abs{B_0}^{2-N}]
        = \abs{x}^{2-N}
        $$
        が成り立つので,$\sup_{t\ge0}E[\abs{B_t}^{2-N}]<\infty$ より $\abs{B}^{2-N}$ は $L^1$-bdd.
        よってThm. 3.19より $\abs{B_{\infty}}^{2-N}$ がa.s.で存在するので,$\lim_{t\to\infty}\abs{B_t}$ がa.s.で存在する.
        a.s.で $\limsup_{t\to\infty}B_t^1=\infty$ となるので,a.s.で $\lim_{t\to\infty}\abs{B_t}=\infty.$
    \end{proof}
    
    \item
    $N=3$ を仮定する.
    Gaussian densityの形式を用いて,r.v.の族 $(\abs{B_t}^{-1})_{t\ge0}$ が $L^2$-bdd.であることを確かめよ.
    また $(\abs{B_t}^{-1})_{t\ge0}$ がCLMであり,かつtrue martingaleでないことを示せ.
    \begin{proof}
        まず $(\abs{B_t}^{-1})_{t\ge0}$ が $L^1$-bdd.であることを示す.
        $\delta:=\frac{\abs{x}}{2}(>0)$ と定める.
        このとき
        \begin{align}
            E[\abs{B_t}^{-2}]
            &= \int_{\real^3}\abs{y}^{-2}p(y)dy
            = \int_{\real^3}\frac{1}{\abs{y}^2(2\pi t)^{3/2}}\exp{\left(-\frac{\abs{y-x}^2}{2t}\right)}dy \\
            &= \int_{\abs{y}<\delta}\frac{1}{\abs{y}^2(2\pi t)^{3/2}}\exp{\left(-\frac{\abs{y-x}^2}{2t}\right)}dy
            + \int_{\abs{y}\ge\delta}\frac{1}{\abs{y}^2(2\pi t)^{3/2}}\exp{\left(-\frac{\abs{y-x}^2}{2t}\right)}dy \\
            &=: I_1(t) + I_2(t).
        \end{align}

        任意の $t>0$ に対し
        \begin{align}
            I_2(t)
            \le \frac{1}{\delta^2}\int_{\real^3}\frac{1}{(2\pi t)^{3/2}}\exp{\left(-\frac{\abs{y-x}^2}{2t}\right)}dy
            \le \frac{1}{\delta^2}
        \end{align}
        と評価できることから $I_2$ は有界.
        あとは $I_1(t)$ が任意の $t>0$ において有界であることを示せば十分.
        $\abs{y}<\delta=\frac{\abs{x}}{2}\implies \abs{y-x}\ge\abs{x}-\abs{y}>\frac{\abs{x}}{2}$ であることより
        \begin{align}
            I_1(t)
            \le \frac{1}{(2\pi t)^{3/2}}\exp{\left(-\frac{\abs{x}^2}{8t}\right)}\int_{\real^3}\frac{1}{\abs{y}^2}dy
            = \frac{1}{(2\pi t)^{3/2}}\exp{\left(-\frac{\abs{x}^2}{8t}\right)}\cdot4\pi\delta
        \end{align}
        が成り立つ.
        $$
        \varphi(t)
        = \frac{1}{(2\pi t)^{3/2}}\exp{\left(-\frac{\abs{x}^2}{8t}\right)}
        $$
        で定めると $\varphi$ は $(0,\infty)$ 上連続かつ $\lim_{t\downarrow0+}\varphi(t)=0, \lim_{t\to\infty}\varphi(t)=0.$
        よって $\varphi$ は有界なのである $M>0$ が存在し,$\sup_{t>0}\abs{\varphi(t)}\le M<\infty.$
        よって $\sup_{t>0}I_1(t)\le 4\pi M\delta$ と評価できるので $I_1$ も有界.
        ゆえに $(\abs{B_t}^{-1})_{t\ge0}$ は $L^2$-bdd.

        次に $(\abs{B_t}^{-1})_{t\ge0}$ がCLMであるがtrue martingaleでないことを示す.
        7. の証明よりCLMであることが言えるので $(\abs{B_t}^{-1})_{t\ge0}$ がtrue martingaleであると仮定すると $L^2$-bdd. martingale.
        a.s.で $\lim_{t\to\infty}\abs{B_t}=\infty(\implies \abs{B_\infty}^{-1}=0)$ とThm. 4.13を合わせて
        $$
        0
        = E[\abs{B_\infty}^{-2}]
        = E[\abs{B_0}^{-2}]
        + E[\gen{\abs{B}^{-1}, \abs{B}^{-1}}_\infty]
        > 0
        $$
        がわかるが,これは矛盾.
        よって $(\abs{B_t}^{-1})_{t\ge0}$ はCLMであるがtrue martingaleでない.
    \end{proof}
\end{enumerate}
\end{document}